
\documentclass[12pt]{article} % Default font size is 12pt, it can be changed here

\usepackage[margin=1.0in]{geometry} % Required to change the page size to A4
\geometry{letterpaper} % Set the page size to be A4 as opposed to the default US Letter

\usepackage{amsmath}

% \title{\vspace{-20mm}XNOR-Net Training\vspace{-10mm}}
% \date{}

\begin{document}
% \maketitle

\section{Backward Gradient of Scaled Sign Function} % Major section
\subsection{Forward scaled sign function}
Assume we have the original weights $\mathbf{W}$ as

\begin{equation}
	\mathbf{W}=[W_{1},W_{2}, ..., W_{n}]
\end{equation}

In the forward pass, we need to binarize $\mathbf{W}$ to $\mathbf{\widetilde{W}}$

\begin{equation}
	\mathbf{\widetilde{W}}=[\widetilde{W}_{1},\widetilde{W}_{2}, ..., \widetilde{W}_{n}]
\end{equation}

Based on the XNOR-Net paper, the binarized weights $\mathbf{\widetilde{W}}$ is calculated by

\begin{equation}
	\mathbf{\widetilde{W}}=\alpha \cdot sign(\mathbf{W})
\end{equation}
\begin{equation}
	\alpha = \frac{1}{n} \cdot {||\mathbf{W}||}_{L1} = \frac{1}{n} \cdot
	\sum_{i=1}^{n} W_{i}
\end{equation}

which means, for each $\widetilde{W}_{i}$, we have

\begin{equation}
	\widetilde{W}_{i} =\alpha \cdot sign(W_{i}), \; i \in \{1,2,...,n\}
\end{equation}

this function is named as the scaled sign function.


\subsection{Original backward gradient of scaled sign function}
In the XNOR-Net paper, assuming $C$ is the cost function, the backward gradient of scaled sign function is given as

\begin{equation}
	\frac{\partial C}{\partial W_{i}} = 
	\frac{\partial C}{\partial \widetilde{W}_{i}}
	(\frac{1}{n}+\frac{\partial sign(W_{i})}{\partial W_{i}} \cdot \alpha)
\end{equation}

which is calculated from

\begin{equation}
	\frac{\partial C}{\partial W_{i}} = 
	\frac{\partial C}{\partial \widetilde{W}_{i}} \cdot
	\frac{\partial \widetilde{W}_{i}}{\partial W_{i}}
\end{equation}

The detailed steps are:

\begin{equation}
\begin{aligned}
	\frac{\partial C}{\partial W_{i}}
	&= 
		\frac{\partial C}{\partial \widetilde{W}_{i}} \cdot
		\frac{\partial \widetilde{W}_{i}}{\partial W_{i}}\\
	&=
		\frac{\partial C}{\partial \widetilde{W}_{i}} \cdot
		\frac{\partial (\alpha \cdot sign(W_{i}))}{\partial W_{i}}\\
	&=
		\frac{\partial C}{\partial \widetilde{W}_{i}} \cdot
		[sign(W_{i})\cdot\frac{\partial \alpha}{\partial W_{i}} + 
		\alpha \cdot \frac{\partial sign(W_{i})}{\partial W_{i}}]\\
	&=
		\frac{\partial C}{\partial \widetilde{W}_{i}}
		[\frac{1}{n}+\frac{\partial sign(W_{i})}{\partial W_{i}} \cdot \alpha]
\end{aligned}
\end{equation}

\subsection{Correct backward gradient}
However, the equation

\begin{equation}
	\frac{\partial C}{\partial W_{i}} = 
	\frac{\partial C}{\partial \widetilde{W}_{i}} \cdot
	\frac{\partial \widetilde{W}_{i}}{\partial W_{i}}
\end{equation}

is actually inaccurate. The correct equation should be

\begin{equation}
\begin{aligned}
	\frac{\partial C}{\partial W_{i}} = 
	\sum_{j=1}^{n} (\frac{\partial C}{\partial \widetilde{W}_j}\cdot
	\frac{\partial \widetilde{W}_j}{\partial W_i})
\end{aligned}
\end{equation}

Therefore, the \textbf{correct backward gradient} should be

\begin{equation}
\begin{aligned}
	\frac{\partial C}{\partial W_{i}} &= 
	\frac{1}{n} \cdot sign(W_{i}) \cdot 
	\sum_{j=1}^{n}[\frac{\partial C}{\partial \widetilde{W}_j} \cdot sign(W_j)]\\
	&\quad +
	\frac{\partial C}{\partial \widetilde{W}_i} \cdot
	\frac{sign(W_i)}{W_i} \cdot \alpha
\end{aligned}
\end{equation}

The detailed steps to get this gradient are

\begin{equation}
\begin{aligned}
	\frac{\partial C}{\partial W_{i}}
	&= 
	\sum_{j=1}^{n} (\frac{\partial C}{\partial \widetilde{W}_j}\cdot
	\frac{\partial \widetilde{W}_j}{\partial W_i}) 
	\\ &=
	\sum_{j=1}^{n} [\frac{\partial C}{\partial \widetilde{W}_j}\cdot
	\frac{\partial (\alpha \cdot sign(W_j))}{\partial W_i}] 
	\\ &=
	\sum_{j=1}^{n} [\frac{\partial C}{\partial \widetilde{W}_j}\cdot
	(sign(W_j)\cdot\frac{\partial \alpha}{\partial W_i} + 
	\frac{\partial sign(W_j)}{\partial W_i}\cdot\alpha)]
	\\ &=
	\sum_{j=1}^{n} [\frac{\partial C}{\partial \widetilde{W}_j}\cdot
	sign(W_j)\cdot\frac{\partial \alpha}{\partial W_i}] + 
	\frac{\partial C}{\partial \widetilde{W}_i} \cdot
	\frac{\partial sign(W_j)}{\partial W_i}\cdot\alpha
	\\ &=
	\frac{\partial \alpha}{\partial W_i}\cdot
	\sum_{j=1}^{n} [\frac{\partial C}{\partial \widetilde{W}_j}\cdot
	sign(W_j)] + 
	\frac{\partial C}{\partial \widetilde{W}_i} \cdot
	\frac{\partial sign(W_j)}{\partial W_i}\cdot\alpha
	\\ &=
	\frac{1}{n}\cdot sign(W_i) \cdot
	\sum_{j=1}^{n} [\frac{\partial C}{\partial \widetilde{W}_j}\cdot
	sign(W_j)] + 
	\frac{\partial C}{\partial \widetilde{W}_i} \cdot
	\frac{\partial sign(W_j)}{\partial W_i}\cdot\alpha
\end{aligned}
\end{equation}


\end{document}
